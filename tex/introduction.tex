% !TeX spellcheck = en_GB
\chapter{Introduction}\label{kap:introduction}
The genetic profile of a population does not only just describe its characteristics,
it also provides insights into its hereditary and evolutionary history.
In eukaryotes, including humans, the evolutionary histories of genes are largely intertwined due to recombination and sexual reproduction.
This contrasts with prokaryotic populations, where genes often follow a dominant clonal lineage \cite{Human_Popgen_Lohmueller_2021}.
However, \ac{HGT} can disrupt this pattern by allowing genes to move between lineages and individuals, introducing genetic variation \cite{Gen_Molekbio_Schmidt_2023}.

The \ac{GFS}, which describes the number of genes observed in $k$ individuals within a population, reflects these evolutionary effects.
Under conditions of clonal inheritance, the \ac{GFS} responds to the underlying tree structure, in which genes are passed on to descendants and thus accumulate within a subtree \cite{Baumdicker_2014}.
Deviations between the observed \ac{GFS} and the \ac{GFS} expected under a neutral clonal model may indicate evolutionary processes beyond simple mutations, indicating possible \ac{HGT}.
Understanding the dynamics of \ac{HGT} is usefull for several reasons.
Firstly, it improves our understanding of microbial evolution, including the role lateral gene flow plays in achieving genetic diversity.
Unlike the slow, gradual accumulation of mutations in a purely clonal model, \ac{HGT} can introduce substantial genetic changes in a short time frame, accelerating adaptation and evolution.
Secondly, horizontal gene transfer has profound implications for public health, particularly with regard to antibiotic resistance.
Resistance-conferring genes can spread rapidly through bacterial populations via \ac{HGT}, undermining treatment efforts and leading to the emergence of multi-drug resistant strains \cite{HGT_Burmeister_2015}.

Methodologically, integrating \ac{HGT} into existing models of genetic evolution improves their accuracy and predictive power.
Traditional models based on clonal inheritance often fail to capture the complexity of microbial evolution.

This thesis provides a comprehensive biological and mathematical background for all concepts used and
extends the open source software \textit{msprime} and \textit{tskit} \cite{Msprime_Baumdicker_2022}.
Specifically, the mutation model of \textit{msprime} is extended by introducing a computational approach of the infinite gene model \cite{Baumdicker_2014} through a gene gain and loss model.
This model allows the simulation of neutral evolution.
Using this model, a neutrality test is developed to determine whether the gene frequency spectrum results from neutral evolutionary processes.
Furthermore, the ancestry and mutation simulation is extended by the introduction of horizontal gene transfer events and a tree fixation mechanism.
The fixation allows the simulation of gene conversion, recombination and \ac{HGT} while maintaining a given tree structure as a backbone.
With that, the effect of gene conversion and \ac{HGT} on the \ac{GFS} is assessed.
Finally, we estimate the strength of \ac{HGT} by recording simulation parameters across different bacterial species which minimises the error between the simulated and real Gene Frequency Spectrum.
This analysis uses pangenome data from the NCBI, allowing for a detailed comparison between simulated results and observed genetic patterns.